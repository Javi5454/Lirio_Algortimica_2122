\documentclass[10pt,a4paper]{article}
\usepackage[utf8]{inputenc}
\usepackage[spanish]{babel}
\usepackage{amsmath}
\usepackage{amsfonts}
\usepackage{amssymb}
\usepackage{graphics}
\usepackage{graphicx}
\usepackage{xcolor}
\usepackage{listings}
\usepackage{csvsimple}
\usepackage{caption}
\usepackage{subcaption}
\usepackage[left=2cm,right=2cm,top=2cm,bottom=2cm]{geometry}

\renewcommand*\contentsname{Índice} %Nombre del indice

\begin{document}
\lstset{
	basicstyle=\footnotesize,
	extendedchars=true,
	literate={á}{{\'a}}1 {ã}{{\~a}}1 {é}{{\'e}}1 {ú}{{\'u}}1 {ó}{{\'o}}1,
	backgroundcolor=\color{black!5}
	}
	
\begin{titlepage}
	\centering
	{\includegraphics[scale=0.5]{Logo_UGR.png}\par}
	\vspace{1cm}
	{\bfseries\Large Escuela T\'ecnica Superior de Ingeniería Informática y Telecomunicaciones \par}
	\vspace{2.5cm}
	{\scshape\Huge Pr\'actica 2: Divide y Vencerás \par}
	\vspace{3cm}
	{\itshape\Large Doble Grado Ingeniería Informática y Matemáticas}
	\vfill
	{\Large Autores: \par}
	{\Large Jose Alberto Hoces Castro\par}
	{\Large Javier Gómez López \par}
	{\Large Moya Mart\'in Castaño \par}
	\vfill
	{\Large Abril 2022 \par}
\end{titlepage}

\thispagestyle{empty}
\null
\vfill

%%Información sobre la licencia
\parbox[t]{\textwidth}{
  \includegraphics[scale=0.05]{by-nc-sa.png}\\[4pt]
  \raggedright % Texto alineado a la izquierda
  \sffamily\large
  {\Large Este trabajo se distribuye bajo una licencia CC BY-NC-SA 4.0.}\\[4pt]
  Eres libre de distribuir y adaptar el material siempre que reconozcas a los\\
  autores originales del documento, no lo utilices para fines comerciales\\
  y lo distribuyas bajo la misma licencia.\\[4pt]
  \texttt{creativecommons.org/licenses/by-nc-sa/4.0/}
}

\newpage

\tableofcontents

\newpage
\section{Introducción}

El objetivo de esta práctica es utilizar la técnica ``divide y vencerás'' para resolver problemas de forma más eficiente que otras alternativas más sencillas o directas. Para ello, se plantean los siguientes dos problemas:

\begin{itemize}
	\item \textbf{Ejercicio 1:} Este problema consiste en realizar la búsqueda de un elemento en un vector ordenado con \(n\) elementos.
	\item \textbf{Ejercicio 2:} Este problema consiste en dados \(k\) vectores de \(n\) elementos, todos ellos ordenados de menor a mayor, combinar todos los vectores en uno único ordenado.
\end{itemize}

\section{Desarrollo}

\subsection{Ejercicio 1}
El enunciado del problema es el siguiente: \textit{Dado un vector ordenado (de forma no decreciente) de números enteros \(v\), todos distintos, el objetivo es determinar si existe un índice \(i\) tal que \(v[i] = i\) y encontrarlo en ese caso. Diseñar e implementar un algoritmo ``divide y vencerás'' que permita resolver el problema. ¿Cuál es la complejidad de ese algoritmo y la del algoritmo ``obvio'' para realizar esta tarea? Realizar también un estudio empírico e híbrido de la eficiencia de ambos algoritmos.}

\textit{Supóngase ahora que los enteros no tienen por qué ser todos distintos (pueden repetirse). Determinar si el algoritmo anterior sigue siendo válido, y en caso negativo proponer uno que sí lo sea. ¿Sigue siendo preferible al algoritmo obvio?}

\subsubsection{Algoritmo ``obvio'' o de fuerza bruta}
La manera obvia de resolver este ejercicio sería mediante un algoritmo secuencial, que vaya recorriendo el vector hasta encontrar el elemento buscado. 

\end{document}