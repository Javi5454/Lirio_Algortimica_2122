% --------------------------------------------------
%  TALLER DE INTRODUCCIÓN A LaTeX
%  https://github.com/mianfg/latex-intro
%
%  Sesión 1 -> Presentación
%
%  Autor: Miguel Ángel Fernández Gutiérrez, @mianfg
%  Fecha: 20 febrero, 2019
% --------------------------------------------------

% Tipo de documento (presentación)
\documentclass[10pt, xcolor=table]{beamer}
\usepackage{caption}
\usepackage{subcaption}

% Cargar el tema
\usetheme{metropolis}

%  __________
% |          |
% | Paquetes |
% |__________|

% Paquetes de idioma
\usepackage[utf8]{inputenc}
\usepackage[spanish, es-tabla, es-lcroman, es-noquoting]{babel}

% Paquete para código fuente
% LISTINGS
\usepackage{listings}
\usepackage{lipsum}
\usepackage{courier}
\usepackage{csvsimple}

% Colores para los bloques de código
\definecolor{codegreen}{rgb}{0,0.6,0}
\definecolor{codegray}{rgb}{0.5,0.5,0.5}
\definecolor{codepurple}{rgb}{0.58,0,0.82}
\definecolor{backcolour}{rgb}{0.95,0.95,0.92}
\lstdefinestyle{mystyle}{
	backgroundcolor=\color{backcolour},   
	commentstyle=\color{codegreen},
	keywordstyle=\color{blue},
	numberstyle=\tiny\color{codegray},
	stringstyle=\color{codepurple},
	basicstyle=\footnotesize\ttfamily,
	breakatwhitespace=false,         
	breaklines=true,                 
	captionpos=b,                    
	keepspaces=true,                 
	numbers=left,                    
	numbersep=5pt,                  
	showspaces=false,                
	showstringspaces=false,
	showtabs=false,                  
	tabsize=4
}
\lstset{style=mystyle}

% Paquete de numeración en Beamer
\usepackage{appendixnumberbeamer}

% Paquete de uso para plantilla
\usepackage{booktabs}
\usepackage[scale=2]{ccicons}

% Paquete para controlar espacios
\usepackage{xspace}
\newcommand{\themename}{\textbf{\textsc{metropolis}}\xspace}

% Paquetes para matemáticas
\usepackage{amsmath}    % Paquete básico de matemáticas
\usepackage{amsthm}     % Teoremas
\usepackage{mathrsfs}   % Fuente para ciertas letras utilizadas en matemáticas

% Paquetes para fuentes
\usepackage{newpxtext, newpxmath}   % Fuente similar a Palatino
\usepackage{FiraSans}               % Fuente sans serif
\usepackage[T1]{fontenc}
\usepackage[italic]{mathastext}     % Utiliza la fuente del documento
                                    % en los entornos matemáticos

%  ________________________
% |                        |
% | Configuración del tema |
% |________________________|

% Configuración básica del tema
\metroset{
  % tema oscuro ('dark') o claro ('light'). No tiene efecto al usar la
  % paleta de colores más adelante
  background=light,
  % 'none' para eliminar la diapositiva inicial de cada sección
  sectionpage=progressbar,
  % 'progressbar' o 'simple' para añadir una diapositiva inicial a cada subsección
  subsectionpage=none,
  % contador de página: 'none', 'counter' o 'fraction'
  numbering=none,
  % barra de progreso: 'none', 'head', 'frametitle' o 'foot'
  progressbar=frametitle,
  % fondo de los bloques estilo teorema: 'transparent' o 'fill'
  block=fill,
}

% Paleta de colores
\definecolor{accent}{HTML}{009688}
\colorlet{darkaccent}{accent!70!black}
\definecolor{foreground}{RGB}{0, 0, 0}
\definecolor{background}{RGB}{255, 255, 255}

% Insertar los colores en el tema
\setbeamercolor{normal text}{fg=foreground, bg=background}
\setbeamercolor{alerted text}{fg=darkaccent, bg=background}
\setbeamercolor{example text}{fg=foreground, bg=background}
\setbeamercolor{frametitle}{fg=background, bg=accent}

\setbeamercolor{headtitle}{fg=background!70!accent,bg=accent!90!foreground}
\setbeamercolor{headnav}{fg=background,bg=accent!90!foreground}
\setbeamercolor{section in head/foot}{fg=background,bg=accent}

\defbeamertemplate*{headline}{miniframes theme no subsection}{
  % Caja para mostrar título y autor encima de cada diapositiva
  % Nosotros no 
  %% \begin{beamercolorbox}[ht=2.5ex,dp=1.125ex,
  %%     leftskip=.3cm,rightskip=.3cm plus1fil]{headtitle}
  %%   {\usebeamerfont{title in head/foot}\insertshorttitle}
  %%   \hfill
  %%   \leavevmode{\usebeamerfont{author in head/foot}\insertshortauthor}
  %% \end{beamercolorbox}
  %% \begin{beamercolorbox}[colsep=1.5pt]{upper separation line head}
  %% \end{beamercolorbox}

  % Caja para mostrar navegación encima de cada diapositiva
  \begin{beamercolorbox}{headnav}
    \vskip2pt\insertnavigation{\paperwidth}\vskip2pt
  \end{beamercolorbox}
  \begin{beamercolorbox}[colsep=1.5pt]{lower separation line head}
  \end{beamercolorbox}
}

%  _________
% |         |
% | Ajustes |
% |_________|

% Fijar tabla a posición
\usepackage{array}
\newcolumntype{L}[1]{>{\raggedright\let\newline\\\arraybackslash\hspace{0pt}}m{#1}}
\newcolumntype{C}[1]{>{\centering\let\newline\\\arraybackslash\hspace{0pt}}m{#1}}
\newcolumntype{R}[1]{>{\raggedleft\let\newline\\\arraybackslash\hspace{0pt}}m{#1}}

%  ________
% |        |
% | Título |
% |________|

\title{Algoritmos Greedy (o voraces)}
\subtitle{Algorítmica. \alert{Práctica 3}}
\date{Mayo 2022}
\author{Jose Alberto Hoces Castro\\Javier Gómez López\\ Manuel Moya Martín Castaño\\[4pt]}
\titlegraphic{\hfill\includegraphics[width=2.5cm]{logo_dark.jpg}}

%  ___________
% |           |
% | Documento |
% |___________|

\begin{document}
\maketitle

\begin{frame}{Contenidos}
	\setbeamertemplate{section in toc}[sections numbered]
	\tableofcontents[]
\end{frame}

\begin{frame}[fragile]{Objetivo de la práctica}
	Aprender a analizar un problema y resolverlo mediante la técnica Greedy, además de justificar su utilidad para resolver problemas de forma muy eficiente, obteniendo la solución óptima o muy cercana a la óptima.
\end{frame}

\section{Ejercicio 1. Contenedores}
\begin{frame}[fragile]{Enunciado}
	\textit{Se tiene un buque mercante cuya capacidad de carga es de } K \textit{toneladas y un conjunto de contenedores \(c_1, \dotsc, c_n\) cuyos pesos respectivos son \(p_1, \dotsc, p_n\) (expresados también en toneladas). Teniendo en cuenta que la capacidad del buque es menor que la suma total de los pesos de los contenedores: }
\end{frame}

\begin{frame}[fragile]{Primer ejercicio}
	\textit{Diseñe un algoritmo que maximice el número de contenedores cargados, y demuestre su optimalidad.}
\end{frame}

\begin{frame}[fragile]{Primer ejercicio. \normalfont{Planteamiento del algoritmo}}
	\begin{itemize}
		\item Como queremos cargar el máximo número de contenedores, empezaremos cargando los más \textbf{pequeños}.
		\item Ordenamos de \textbf{menor a mayor} peso los contenedores.
		\item Empezamos a cargar los de menor peso hasta que superemos las K toneladas del buque mercante.
		\item Todo esto lo simulamos con un vector de enteros en nuestro código, el cual tenemos a continuación.
	\end{itemize}
\end{frame}

\begin{frame}[fragile]{Primer ejercicio. \normalfont{Código}}
	\lstinputlisting[language=C++]{./Codes/contenedores1.cpp}
\end{frame}

\begin{frame}[fragile]{Primer ejercicio. \normalfont{Enfoque Greedy}}
	Las 6 características de nuestro problema que hacen que lo identifiquemos como problema Greedy son:
	\begin{itemize}
		\item \textbf{Un conjunto de candidatos}: En este caso, los contenedores a cargar.
		\item \textbf{Una lista de candidatos ya usados}: Los contenedores que ya han sido cargados.
		\item \textbf{Un criterio que dice cuándo un conjunto de candidatos forma una solución}: El criterio es que la suma de los pesos de un conjunto de contenedores no sea superior a las K toneladas del buque.
	\end{itemize}
	
\end{frame}

\begin{frame}[fragile]{Primer ejercicio. \normalfont{Enfoque Greedy}}
	\begin{itemize}
		\item \textbf{Un criterio que dice cuándo un conjunto de candidatos es factible (podrá llegar a ser una solución)}: el conjunto de contenedores que se evalúe no debe superar en peso las K toneladas del buque.
		\item \textbf{Una función de selección que indica en cualquier instante cuál es el candidato más prometedor de los no usados todavía}: El contenedor de menor peso de los que aún no están cargados, de ahí que los ordenemos de menor a mayor peso.
		\item \textbf{La función objetivo que intentamos optimizar}: El número de contenedores a cargar, es lo que queremos maximizar.
	\end{itemize}
	
\end{frame}

\begin{frame}[fragile]{Primer ejercicio. \normalfont{Estudio de la optimalidad}}
\end{frame}
	
\begin{frame}[fragile]{Segundo ejercicio}
	\textit{Diseñe un algoritmo que intente maximizar el número de toneladas cargadas.}
\end{frame}

\begin{frame}[fragile]{Segundo ejercicio. \normalfont{Planteamiento del algoritmo}}
	\begin{itemize}
		\item Como queremos cargar el máximo número de toneladas, empezaremos cargando los más \textbf{pesados}.
		\item Ordenamos de \textbf{mayor a menor} peso los contenedores.
		\item Empezamos a cargar los de mayor peso hasta que superemos las K toneladas del buque mercante.
		\item Todo esto lo simulamos con un vector de enteros en nuestro código, el cual tenemos a continuación.
	\end{itemize}
\end{frame}

\begin{frame}[fragile]{Segundo ejercicio. \normalfont{Código}}
	\lstinputlisting[language=C++]{./Codes/contenedores2.cpp}
\end{frame}

\begin{frame}[fragile]{Segundo ejercicio. \normalfont{Estudio de la optimalidad}}
	\centering [5, 4, 6, 1, 1, 2, 7, 9, 8, 3] \hspace{0.2cm}K = 10
	\\
	\centering $\downarrow$
	\\
	\centering [9, 8, 7, 6, 5, 4, 3, 2, 1, 1]
	\\
	\textbf{Solución aportada por nuestro algoritmo}: [9]
	\\
	\textbf{Solución óptima}: [1,2,3,4]
	
\end{frame}
\end{document}