\documentclass[10pt,a4paper]{article}
\usepackage[utf8]{inputenc}
\usepackage[spanish]{babel}
\usepackage{amsmath}
\usepackage{amsfonts}
\usepackage{amssymb}
\usepackage{graphics}
\usepackage{graphicx}
\usepackage{xcolor}
\usepackage{listings}
\usepackage{csvsimple}
\usepackage{caption}
\usepackage{subcaption}
\usepackage[left=2cm,right=2cm,top=2cm,bottom=2cm]{geometry}

\renewcommand*\contentsname{Índice} %Nombre del indice

\begin{document}
\lstset{
	basicstyle=\footnotesize,
	extendedchars=true,
	literate={á}{{\'a}}1 {ã}{{\~a}}1 {é}{{\'e}}1 {ú}{{\'u}}1 {ó}{{\'o}}1,
	backgroundcolor=\color{black!5}
	}
	
\begin{titlepage}
	\centering
	{\includegraphics[scale=0.5]{Logo_UGR.png}\par}
	\vspace{1cm}
	{\bfseries\Large Escuela T\'ecnica Superior de Ingeniería Informática y Telecomunicaciones \par}
	\vspace{2.5cm}
	{\scshape\Huge Pr\'actica 3: Algoritmos Greedy \par}
	\vspace{3cm}
	{\itshape\Large Doble Grado Ingeniería Informática y Matemáticas}
	\vfill
	{\Large Autores: \par}
	{\Large Jose Alberto Hoces Castro\par}
	{\Large Javier Gómez López \par}
	{\Large Moya Mart\'in Castaño \par}
	\vfill
	{\Large Mayo 2022 \par}
\end{titlepage}

\thispagestyle{empty}
\null
\vfill

%%Información sobre la licencia
\parbox[t]{\textwidth}{
  \includegraphics[scale=0.05]{by-nc-sa.png}\\[4pt]
  \raggedright % Texto alineado a la izquierda
  \sffamily\large
  {\Large Este trabajo se distribuye bajo una licencia CC BY-NC-SA 4.0.}\\[4pt]
  Eres libre de distribuir y adaptar el material siempre que reconozcas a los\\
  autores originales del documento, no lo utilices para fines comerciales\\
  y lo distribuyas bajo la misma licencia.\\[4pt]
  \texttt{creativecommons.org/licenses/by-nc-sa/4.0/}
}

\newpage

\tableofcontents

\newpage

\section{Introducción}

El objetivo de esta práctica es aprender a implementar y utilizar algoritmos \textit{``greedy''} o voraces para resolver problemas de manera rápida aunque no por ello menos óptima. Para ello, se plantean los siguientes dos problemas:

\begin{itemize}
	\item \textbf{Ejercicio 1} (Contenedores): Se quiere rellenar un buque mercante con una cierta capacidad de peso con contenedores, cada uno de los cuales tiene su propio peso.
	\item \textbf{Ejercicio 2} (TSP): El problema del viajero. Se quiere recorrer una serie de ciudades, pasando por ellas solo una vez y volviendo al punto de partida. Se quiere encontrar la ruta más óptima.
\end{itemize}

\section{Desarrollo}

Para el análisis de los algoritmos que desarrollaremos, hemos realizado los siguientes pasos:

\begin{enumerate}
	\item Un \textbf{análisis teórico} de los algoritmos usando las técnicas vistas en clase.
	
	\item Un \textbf{análisis empírico} donde hemos ejecutado los algoritmos en nuestros ordenadores bajo las mismas normas y condiciones. Hemos compilado usando la optimización \texttt{-Og}. Además, hemos usado como \textit{datasets} de pruebas los datos proporcionados por la profesora en el caso del TSP, y valores aleatorios de los pesos de los contenedores para el primer problema. Por otro lado, para automatizar el proceso, hemos creado unos \textit{scripts} de generació  de datos de prueba y de ejecución de nuestros programas. Hemos ejecutado cada algoritmo 15 veces en cada uno de los tamaños probados, y hemos hecho la media de ellos para reducir perturbaciones que puedan alterar el resultado.
	
	\item Un \textbf{análisis híbrido} donde hemos tomado los datos de cada uno de los alumnos del grupo y hemos hallado la \(K\) (constante oculta). Para ello hemos usado gnuplot.
\end{enumerate}

\subsection{Ejercicio 1. Contenedores}

El enunciado del problema es el siguiente: \textit{Se tiene un buque mercante cuya capacidad de carga es de } K \textit{toneladas y un conjunto de contenedores \(c_1, \dotsc, c_n\) cuyos pesos respectivos son \(p_1, \dotsc, p_n\) (expresados también en toneladas). Teniendo en cuenta que la capacidad del buque es menor que la suma total de los pesos de los contenedores: }

\begin{itemize}
	\item \textit{Diseñe un algoritmo que maximice el número de contenedores cargados, y demuestre su optimalidad.}
	\item \textit{Diseñe un algoritmo que intente maximizar el número de toneladas cargadas.}
\end{itemize}


\end{document}